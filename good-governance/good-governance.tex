\documentclass[11pt, openany]{book}

%PACKAGES%
\usepackage[inner=0.75in, outer=1in, top=0.75in, bottom=1in, a5paper]{geometry}
\usepackage{graphicx}
\usepackage{fontspec}
%\usepackage{letterspace}
\usepackage{sectsty}
\usepackage{titlesec}
\usepackage{soul}
\usepackage{verse}
\usepackage{fix-cm}%font size
\usepackage{lettrine}
\usepackage[hyphens]{url}
\usepackage[unicode, hidelinks, pdfauthor={Ajahn Brahm}, pdftitle={Good Governance}, pdfsubject={Buddhism}, pdfkeywords={Buddhism, governance}, pdfcreator={Luatex}]{hyperref} %ADDS METADATA%
%\usepackage{pagegrid}
%PACKAGES%
%\pagegridsetup{top-left, step=3.435in}



%MICROTYPOGRAPHY%
\usepackage{microtype}
\frenchspacing
%MICROTYPOGRAPHY%



%LINESPACE%
\usepackage{setspace}
\setstretch{1.12}
\setlength{\parskip}{0pt}
%LINESPACE%

%VERSE%
\settowidth{\versewidth}
{mmmmmmmmmmmmmmmmmmm}%THIS SETS THE GLOBAL DEFAULT WIDTH OF CENTERING. IT IS USUALLY DETERMINED LOCALLY, HOWEVER.%
%VERSE%

%HEADER%
\usepackage{fancyhdr}
\setlength{\headheight}{15pt}
\pagestyle{fancy}

\fancyhf{}
\fancyfoot[CE,CO]{– \thepage \hspace{0.18em}–}

\renewcommand{\headrulewidth}{0pt}
\fancypagestyle{plain}{ %
\fancyhf{} % remove everything
\renewcommand{\headrulewidth}{0pt}
\renewcommand{\footrulewidth}{0pt}}
%HEADER%


%FONTS%
\setmainfont[]{Alegreya ht Pro}
\setsansfont[]{Alegreya Sans}
%FONTS%

%HEADINGS%
\titleformat{\chapter}[display]
{\filcenter}
{\Chapnumfont\Large\MakeLowercase{\scshape{\chaptertitlename}} \thechapter}
{1ex}
{\Huge\Chapfont}

\newfontfamily\Chapfont{Alegreya Sans}
\newfontfamily\Chapnumfont[RawFeature=+c2sc]{Alegreya Sans SC Medium}
\newfontfamily\Secfont{Alegreya Sans Bold}
\sectionfont{\Secfont}


%HEADINGS%



%EPIGRAPH%
\newenvironment{epigraph}



%EPIGRAPH%

%HANGING LEFT%
\newcommand*{\vleftofline}[1]{\leavevmode\llap{#1}}
%HANGINGLEFT%

%WIDOWS & ORPHANS%
\widowpenalty=10000
\clubpenalty=10000
%WIDOWS & ORPHANS%






%DOCUMENT INFO. NOT USED IN TEXT.%
\title{Good Governance}
\author{Ajahn Brahm}
\date{}
%DOCUMENT INFO. NOT USED IN TEXT.%

\begin{document}
\frontmatter
\pagestyle{empty}
\newpage
\begin{center}\end{center}
\vspace{6em}
\begin{center}\begin{LARGE}\emph{The Buddhist Contribution to}\end{LARGE}\\
\vspace{1em}
\begin{Huge}\begin{sc}Good Governance\end{sc}\end{Huge}
\vspace{22em}

\begin{huge}\emph{Ajahn Brahm}\end{huge}


\end{center}



\newpage







\begin{small}{\sffamily

\noindent Copyright © Ajahn Brahmavamso 2018

\medskip

\noindent Creative Commons Attribution-NonCommercial-ShareAlike 4.0 International (CC BY-NC-SA 4.0)

\medskip

\noindent You are free to:\\
\textbf{Share} — copy and redistribute the material in any medium or format\\
\textbf{Adapt} — remix, transform, and build upon the material\\

\noindent Under the following terms:

\noindent \textbf{Attribution} — You must give appropriate credit, provide a link to the license, and indicate if changes were made. You may do so in any reasonable manner, but not in any way that suggests the licensor endorses you or your use.\\
\textbf{Non-Commercial} — You may not use the material for commercial purposes.\\
\textbf{ShareAlike} — If you remix, transform, or build upon the material, you must distribute your contributions under the same license as the original.\\
\textbf{No additional restrictions} — You may not apply legal terms or technological measures that legally restrict others from doing anything the license permits.


\vspace{4em}

\noindent For enquiry about the book or Ajahn Brahm, please contact www.bswa.org

}


\end{small}
\mainmatter
\pagestyle{fancy}




\chapter*{Preface}

The Buddhist Contribution to Good Governance by Ajahn Brahm is based on a Keynote Speech that Ajahn Brahm gave to the United Nations 2007 Vesak Celebrations at Buddhamonthon, Nakhonpathom Province, Thailand in May 2007. The lay supporters in Thailand asked Ajahn Brahm for permission to publish the Keynote Speech in a book form for free distribution, which Ajahn Brahm very kindly agreed to.

We take this opportunity to record our profound gratitude to Phra Ajahn for his kind permission.

\chapter*{Preface to this revised edition}

This revised edition is produced not only in print but also in an electronic format in order that this valuable insight will be more readily available and accessible, reaching a wider audience.

The electronic copy is downloadable for free at www.bswa.org and www.wisdomandwonders.org.

\bigskip

\noindent Bodhinyana Volunteer Team\\
\noindent Vesak 2018

\chapter*{Introduction}

The oldest multinational corporation in the world is the Buddhist Saṅgha. As it happens, I have the franchise in Western Australia where I am the ‘managing director’ of Bodhinyana Monastery in Perth and the ‘CEO’ of the Buddhist Society of Western Australia. So, I should know a great deal about Buddhist Good Governance.

Buddhism has been a positive inspiration for many world leaders. Few know that the British statesman and war time Prime Minister Winston Churchill had a statue of the Buddha by his bedside throughout the Second World War, while his wife, Clementine, had a statue of Guan Yin, the Buddhist Goddess of Mercy.

Here, I will discuss the Buddhist contribution to good governance using the Buddhist monastic code and other Buddhist principles to illuminate the following three modern categories of governance:

\begin{enumerate}
\item Leadership skills
\item Decision-making
\item Problem-solving
\end{enumerate}

\chapter{Leadership Skills}

A successful leader is one who inspires others to action: (a) by example, (b) with authority, and (c) through kindness.

\section{Leading by example}

The great political leader Mahatma Gandhi succeeded in liberating his country of India from colonial domination with surprisingly little bloodshed. The leadership skills that he possessed were developed over a lifetime and were essential to his ability to maintain harmony in all the diverging factions of the independence movement. The greatest of his leadership skills, in my view, was his ability to lead by personal example.

When Gandhi was young and studying law in London, he resided in a common boarding house. One day, his landlady asked him for help with her teenage son. The landlady complained that her son was eating too much sugar and that he would not listen to his mother’s advice, but he was in awe of their young Indian tenant, Mr Gandhi. Perhaps, she thought, if Mr. Gandhi would ask her son not to eat so much sugar, then he might stop. Mr Gandhi agreed to speak with his landlady’s son.

A fortnight later, the landlady complained to Mr Gandhi that her son was still eating too much sugar and asked why he had not spoken to him yet. Mr Gandhi replied that he had spoken to her son, but only that morning. When Mr Gandhi was asked by his landlady why he had waited a fortnight to counsel her son, he replied that he had to wait until he himself had given up sugar, which was only yesterday.

A leader who would only ask others to do what they already do themselves is one who leads by example. They will be successful; whereas, the others, the hypocrites, will be ignored.

The Buddha clearly taught that the leaders of the Saṅgha should not be allowed to be preceptors or teachers unless they have sufficient virtue, meditation experience, and wisdom (Vinaya/Khandaka/Mahavagga, Mv. 1.36-37), and that kings should be virtuous and diligent (Cakkavatti-Sīhanāda Sutta, DN 26). In other words, Buddhism extols leading by example.

\section{Leadership with Authority}

There are three types of authority: conferred authority, inherent authority, and assumed authority. Only the first two types have any legitimacy and are thus sustainable.

Conferred authority is where one is elected to govern. Few in the West realize that the earliest democracy was probably in India, not in Greece, as is commonly assumed. In the time of the Buddha, i.e. before the time of Socrates, the Vajjian Republic centred around Vesāli (on the northern side of the Ganges opposite Patna) was already a long-established democracy. All decisions of governance were made by the citizens of the republic meeting in the Sabha (now the name of the modern Indian parliament in New Delhi) by consensus. When a leader was required, for example of the army or the treasury, they would be appointed by the citizens. Decisions would only be made when they were unanimous, which meant that after a vigorous discussion, the minority would concede the point and “agree to disagree” by voting with their opponents. We know this because the rules governing the business of the Buddhist Saṅgha were derived from this ancient democratic model. Such conferred authority was effective because, as the proverb says: “A leader only has the power that others give them”.

Inherent authority is where a person clearly merits authority due to their superior knowledge or exceptional abilities. In the Buddhist Saṅgha, the elders in the monastery, monks and nuns who have been ordained for many years, have inherent authority due to their greater experience. Also, learned monks and nuns, and those with attainments in deep meditation or enlightenment, have inherent authority due to their personal accomplishments. In a government, a company or any other organization, those who have served successfully for a long time and those with outstanding abilities will automatically become leaders through inherent authority.

Assumed authority, the third type discussed here, is where a person or a group seizes governance for themselves through force or contrivance. Since its legitimacy is controversial, it will always be unstable. As the 17th century English poet William Blake wrote:

\begin{verse} [\versewidth]
The hand of Vengeance found the bed\\
To which the purple tyrant fled;\\
The iron hand crush’d the tyrant’s head,\\
And became a tyrant in his stead.
\end{verse}

In the time of the Buddha, Devadatta attempted to assume authority over the Saṅgha and the result was tragic, as is usually the case. The longevity and strength of the Buddhist Saṅgha is in no small part due to its leadership being a combination of conferred authority and inherent authority. Similarly, the strength of the early Roman Empire rested on its governing body, the Senate, combining conferred and inherent authority. Indeed, a Roman citizen who was standing for election to the Senate would wear white clothing as a symbol of their stainless conduct in the past, thus claiming inherent authority. The Latin word for (bright) white is \emph{candidus}, from which we get the word ‘candidate’ or one standing for election. The candidates were subsequently elected to the governing body, thus giving them conferred authority as well.

Buddhism suggests, through its code of monastic rules, that the most effective and sustainable form of leadership is one with a combination of conferred and inherent authority, avoiding all forms of assumed authority.

\section{Leadership Through Kindness}

A general in the Chinese Imperial Army was once asked by the Emperor why his soldiers had such perfect discipline, while other troops had not. He replied that his soldiers would always obey him because he only told them to do what they already wanted to do!

This was not meant as a joke, rather it reveals the secret of effective leadership. The secret is motivation. That general’s troops had been so perfectly motivated by him that when he told them to train long and hard, they already had the desire to train. When he ordered them to march, they loved marching. And when he commanded them into battle at the risk of their very lives, they wanted to rush into the fight. They had perfect discipline because they had been so well motivated. An effective leader is skilful at motivating others.

The most successful method of motivation is through kindness. Buddhism, and modern psychology too, recognises that attempting to motivate another using anger or fear (known as negative reinforcement) only results in temporary cooperation, long-term resentment, and eventual rebellion. Motivating another through kindness (known as positive reinforcement) may be a little slower, but it has been shown again and again to lead to long-term commitment and collaboration.

The managers of Farrelly Facilities and Engineering — an engineering firm in the U.K. that specialised in installing and maintaining heating devices — decided to ban all overtime in their company. Their compassionate strategy won them awards from bodies such as the Best Engineering Services Technology and Small Business Achievers, since over a two-year period after the company made the culture change, their turnover doubled and profits tripled, and the employees described themselves as ‘among the happiest on this planet’. Such success is typical when managers lead through kindness.

A leader is the captain, not the team. They are the front pair of feet of the centipede, and if the other 49 pairs of feet don’t follow, the centipede gets nowhere. Kindness is the compelling force that beckons all others to follow. The Buddha taught in the Kosambiya Sutta (MN 48), that when the monks maintain bodily, verbal, and mental acts of kindness, both in public and in private, then that will generate concord and unity. And without unity, there is little progress. Thus, Buddhism extols leadership through kindness.

\chapter{Decision-Making}

It is the responsibility of a leader to make decisions. Buddhism’s contribution to the psychology of decision-making is profound. The Buddha taught that, before making any decisions, one should first ensure that one is not acting out of the four agati: self-interest, ill will, delusion, or fear.

\section{Self-Interest}

Buddhism teaches that one of the duties of an ideal monarch (Cakkavatti dhammarājā) is to practise generosity. What does this mean? In modern day Buddhism, some people go to the temple to give requisites so that the pretty girl that they like will fall in love with them, or they invite monks to a meal at their office so that their profits will go up, or they give a large monetary donation to a monastery building project so that their name will appear on a big brass plaque. That is not generosity; it is doing a deal. It is a form of business where one gives a sum of money to the temple in return for advertising rights, for one’s own name on the big brass plaque. True giving has to be done expecting nothing back in return.

Once, I received a phone call from a Polish woman asking about a lecture I was to give that very evening:

“How much do you charge?” she enquired.

“Nothing,” I replied.

“No! You don’t understand,” she complained, “How much must I pay to attend the lecture?”

“Madam, you don’t pay anything. It is all for free,” I said calmly.

“Listen to me!” she shouted down the phone, “DOLLARS! CENTS! HOW MUCH DO I HAVE TO COUGH UP TO GET IN?”

“Madam, you do not have to “cough up” (meaning, pay) anything at all. We will not take your name, nor press any literature on you. And if you don’t like the talk, then you may walk out at any time. Really, it is free,” I explained.

There was a long pause, followed by the sincerest of questions, “Well, if it is for free, then what do you guys (here meaning the monks) get out of it then?”

I replied, “Happiness, madam. Only happiness”.”

Similarly, the ideal form of governance in Buddhism would have leaders embracing self-sacrifice and not self-interest. They would lead without any concern for material reward. Their only reward would be in the happiness of service. It would reflect what President John Fitzgerald Kennedy once said: it is not what your country can do for you, but what you can do for country. But it now applies to the leadership, not the citizens.

\section{Ill Will}

A decision motivated by ill will only generates more ill will. As the famous verse in the Dhammapada states that hatred never cease through hatred in this world. The inspirational leader Nelson Mandela was imprisoned unfairly for 26 years. A short time after his release from jail he became president of South Africa with full power to take revenge upon those who imprisoned him. Instead, he forgave. He governed without ill will towards those who had oppressed his race. Because of Mandela’s attitude, the nation was led forward with surprisingly little violence out of the darkness of apartheid.

Mahatma Gandhi was beaten many times, as well as imprisoned. He too led without ill will. He once said, “I can see a thousand reasons for giving my life for a good cause, but not one reason for taking the life of another.” Such leadership drove one of the most powerful empires of its time out from his country. I also wish to quote a Russian proverb made famous by President Yeltsin:

“One cannot sit on a throne made of swords.”

These anecdotes make clear that governing with ill will, hatred, or revenge, only makes the ruler into a despised despot. All progress is obstructed, and the government becomes unstable.

Decisions made from ill will only generate more problems. In Greek mythology there was a many-headed monster called the Hydra. When an assailant cut off one of the Hydra’s heads, the monster would quickly grow another two! A decision motivated by ill will is like cutting off one of the Hydra’s heads, it only doubles the number of problems.

\section{Delusion}

When one is in thrall to self-interest or desire, one only sees what one wants to see. When one is under the influence of ill will, one is in denial to all that isn’t negative. Self-interest and ill will are the first causes of delusion. When a leader is deluded through self-interest or ill will, they will misunderstand the situation and make grave errors in their governance.

Instead, a leader should put aside all self-interest and ill will, and gather as much information as possible before making a decision. Hasty decisions are often wrong decisions.

One of the classic Buddhist parables from the Udāna is a paradigm for good governance. It is the story of the elephant and the seven blind men.

A king many centuries ago had trouble with his ministers. They would argue so much that almost nothing was decided. The ministers, following the most ancient of political traditions, each claimed that they alone were right and everyone else was wrong. However, when the resourceful king organised a special public festival, they all agreed to take the day off.

The festival was a spectacular affair held in the large stadium. There was singing and dancing, acrobatics, clowns, music and much else. Then for the finale in front of the huge crowd, with the ministers occupying the best seats of course, the king himself led his royal elephant into the centre of the arena. Following the elephant came seven blind men, known in the city to have been blind since birth.

The king took the hand of the first blind man, helped him feel the elephant’s trunk and told him that this is an elephant. He then helped the second blind man feel the elephant’s tusk, the third one its ears, the forth the head, the fifth the torso, the sixth the legs, and the seventh the tail, telling each one that this was an elephant. Then he returned to the first blind man and asked him to say in a loud voice what an elephant was.

“In my considered and expert opinion,” said the first blind man, feeling the trunk, “I state with utmost certainty that an ’elephant’ is a species of snake, genus Python Asiaticus.”

“What twittering twaddle!” exclaimed the second blind man, feeling a tusk. “An ‘elephant’ is much too solid to be a snake. In actual fact, and I am never wrong, it is a farmer’s plough.”

“Don’t be ridiculous!” jeered the third blind man, feeling an ear. “An elephant is a palm-leaf fan.”

“You incompetent idiots,” laughed the fourth blind man, feeling the head. “An elephant is obviously a large water jar.”

“Impossible! Absolutely impossible!” ranted the fifth blind man, feeling the torso. “An elephant is a huge rock.”

“Preposterous!” shouted the sixth blind man, feeling a leg. “An elephant is a tree trunk.”

“What a bunch of twerps!” sneered the last blind man, feeling the tail. “I’ll tell you what an elephant really is. It is a kind of fly whisk. I know, I can feel it!”

“Rubbish it’s a snake.”

“Can’t be! It’s a jar.”

“No way! It’s a …”

And the blind men started arguing so heatedly, and all at the same time, that the words melted together in one loud and long yell. As the insults flew, so did the fists. Though the blind men weren’t quite sure who they were hitting, it didn’t seem that important in the fury of the fracas. They were fighting for principle, for integrity, for truth. Their own individual truth, that was.

While the king’s soldiers were separating the blind and bruised brawlers, the crowd in the stadium was mocking the silent, shamefaced ministers. Everyone who was there had well understood the point of the king’s object lesson.

Each one of us can know only a part of the whole that constitutes truth. When we hold on to our limited knowledge as absolute truth, we are like one of the blind men feeling a part of the elephant and inferring that their own partial experience is the truth, all else being wrong. Instead of blind faith, we can have dialogue. Imagine the result if the seven blind men, instead of opposing their data, had combined their experience. They would have concluded that an ‘elephant’ is something like a huge rock standing on four stout tree trunks. On the back of the rock is a fly whisk, on the front a large water jar. At the sides of the jar are two palm-leaf fans, with two ploughs towards the bottom and a long python in the middle! That would not be such a bad description of an elephant, for one who will never see one.

\section{Fear}

Many ineffectual leaders are crippled through fear of what other people think of them. I have advised many government ministers, a premier and even an executive president never allow the media to control your happiness. Many a government can be imprisoned by free press! Timidity is not good governance.

Many years ago, when I was a monk in Thailand wandering on tudong (a Thai derivative of the Pali word ‘dhutaṅga’, used to refer to the austere or ascetic practices allowed by the Buddha for his disciples), I was sitting in meditation in a wild forest in the Northeast during the night. There were tigers and elephants in that forest at that time, but it was rare to see them. Nevertheless, I knew they were there.

My quiet meditation was disturbed by a rustle in the jungle thickets not far away. With cool mindfulness I quickly judged that it was only a small animal a long distance off, and so gave it no more interest. As it came closer, it disturbed my meditation again. As I listened carefully, I soon assessed that I had been wrong before. This was a medium-sized animal, not a small one, and it was approaching the spot where I was sitting. So, I returned to my breath, but now I was a little concerned. As it neared my seat, my mindfulness became sharpened by fear. I listened intently. This was clearly not a small animal, nor even a medium-sized one, but this was a huge animal by the sound, probably a tiger, and it was coming straight towards me! I opened my eyes in fear, expecting to see a monk-devouring tiger a few feet in front of me. But all I saw was a tiny little mouse.

Fear had amplified the sound of a forest mouse into that of a tiger. That is what fear does. It bends the facts into something completely different from the truth. One should never make decisions out of fear.

\chapter{Problem-solving}

A leader’s duty is not only to make decisions avoiding self-interest, ill will, delusion, or fear, but also to be a wise problem-solver. A leader must be skilful in giving feedback, taking feedback, and creating a meaningful agenda.


\section{Giving Feedback}

The leader of a team is responsible for nurturing its
members by giving them feedback. In modern business jargon this is called ‘performance assessment’. A team cannot be successful when it is carrying a member who is either not pulling his weight or downright obstructive. It is the leader’s job to ensure that every member of their team is performing well.

Most leaders think it is unnecessary to give feedback when things are going well. How wrong they are! Also, they are afraid to give negative feedback because they don’t know how to criticise without giving offence. They only know how to intervene when the situation is so bad that all they can say is “You are fired!” It doesn’t need to get to such a sorry state.

Buddhist teachings offer effective guidelines for admonishing another. Though found in the Vinaya (monastic code), they are applicable in all areas of governance.

First, before admonishing another, one must ensure that one is not doing the same mistake oneself, or a similar one. Otherwise, one will be dismissed as a mere hypocrite.

Second, one should be reasonably certain that one has all the facts and is not misconstruing them. To make my point on how easy it is to jump to a wrong conclusion with only some of the facts, I’ll tell the story of how I, many years ago, spent some of the happiest hours of my life in the loving arms of another man’s wife. We hugged, caressed, and kissed. Even though this was with another man’s wife, it wasn’t adultery. You see, that other man was my father and his wife, whom I kissed, was my mother when I was young! So, secondly, admonish only when you have all the facts.

Third, make sure that your admonishment is motivated by good will. An effective way of conveying to the other person that you are coming from good will is to keep to the standard formula of five instances of praise to every instance of criticism. You point out five things that you value about their work before you bring up the failing. That way, the person being admonished will see their failure in perspective. They will not feel rejected and angry. Instead, they will feel appreciated but with room for improvement. This is called positive feedback and psychology. Buddhism knows that when a person feels valued, they are more likely to be open to criticism and do something about it.

Fourth, the criticism should be at the right time and place, certainly never in public where embarrassment throws up an unhelpful barrier, nor where the other person is over-burdened with too many other jobs. We listen more clearly when we are relaxed. The right time is never too long after the event. Sir Alex Ferguson, manager of Manchester United, does not wait until next month to give his players feedback. He doesn’t wait even until after the match but shouts it from the sidelines. Perhaps this is not quite the Buddhist way of good governance, but timely admonition can “nip the problem in the bud”. Moreover, the person receiving the feedback can understand what you are talking about, because it is fresh in everyone’s mind. When admonishment comes too late, you usually hear the response, “I didn’t realise it was a problem! Why on earth didn’t you tell me earlier?” That is not a sign of good governance.

Fifth, when one admonishes another, it is helpful to make clear that it isn’t their problem, nor is it my problem. But it is OUR problem. When the leader and those led share the responsibility for fixing the inadequacies, then there is a very good chance of finding a solution that doesn’t alienate anyone.

This is how Buddhism advises to give feedback.

\section{Taking Feedback}

Venerable Sāri\-putta, one of the Buddha’s chief disciples, was walking to the village for alms round one day when a little novice monk told the great arahant that he was improperly dressed. The status of Venerable Sāri\-putta will be like a fabulously wealthy and powerful vice president of a Fortune 500 company being criticised by an office boy. But Venerable Sāri\-putta never shouted, “Who do you think you are?! You insignificant little mouse? Get out of my sight!”, as would probably have been the case with the VP and the office boy. Instead, Venerable Sāri\-putta looked at his robes, noticed that the novice was right, went behind a thicket to adjust the robe then came out to thank the novice. The great, enlightened Sāri\-putta, second in wisdom only to the Buddha, then called the little novice his teacher. That is how great leaders take feedback.

Or, as Ajahn Chah used to say: if someone calls you a dog, then look at your bottom. If you can’t see a tail there, then you do not need to accept the criticism. You aren’t a dog, so there is no problem. But if you see a tail on your bottom, then thank the one who has admonished you. They were right.

When leaders do not know how to take feedback, given skilfully as described just above, then they are not only obstructing progress in their organisation, but they are also hindering their own personal growth. This is not good governance.

Often, leaders need to go way beyond their inner circle to get honest feedback. Otherwise, the following scenario occurs:

“Three aspiring politicians were being interviewed for selection to stand for the party at the coming election.

When the first person came in, he was asked, “What is 2 plus 2?” Fumbling nervously with his calculator, he answered, “Four”.

The second person was asked the same question and immediately answered, “Four, sir.”

The third, when asked the same question replied, “What is 2 plus 2? It is whatever you tell me it is, boss!” And guess who became the politician?

It takes humility, courage and effort for leaders to seek out feedback from beyond their inner circle, and even beyond their outer circle. But without such feedback, good governance is lost.

\section{Creating a meaningful agenda}

Whether one is a prime minister or a Sangharāja (supreme
 patriarch), an abbot, a CEO, a general or a department manager, one also has the role of steering the country or the organisation towards a worthwhile goal. This is ‘the big picture’ part of governance. I have deliberately left it to the last, because many managers never seem to get past ‘the big picture’. It is the leader’s job to articulate the agenda, convince others of its value, and embody it through example.

In a Buddhist monastery, there is little meaning in our life unless we aim for enlightenment. That has to be our agenda and the abbot must reinforce that goal and even embody it, at least to a degree. In a government of a country, Buddhism would promote an agenda more meaningful than mere economic prosperity. One can only admire the King of Bhutan for making Gross National Happiness the main agenda of his government. In a business, profits and share prices are important, though not meaningful enough for those employees in the company. Many key employees, the talent pool that drives a company’s success, demand job satisfaction or they leave. They only thrive in a work environment where they learn new skills, develop existing abilities, and have an overall positive effect of the society in which they live. In the military, servicemen and servicewomen need to feel that their sacrifices are contributing to a better world; otherwise, their morale dissipates, and they malfunction. Good governance also requires setting a meaningful agenda.

\chapter*{Conclusion — Karma}

Buddhist teachings such as the ideal of the ‘wheel-turning
 righteous leader’ suggest that a better world is possible, and
that it can be achieved through wise and compassionate governance as described above. The Law of Kamma means that we have the power to create happiness or suffering, that we are responsible for our future. Leaders carry a greater share of this responsibility than others.

Some leaders in other religions pass the buck of responsibility to a God or to fate. The Buddhist Law of Kamma returns responsibility to where it belongs, in the hand of the leaders. Even St. Augustine, the giant of early Christian theology, recognised the inefficiency of appealing to a God to help solve the world’s problems when he remarked, “Either God won’t stop suffering or He can’t stop suffering.” So, when a leader prays to a supernatural being in a temple, shrine or church, they are probably wasting precious time that could be better used in grappling with the problems themselves. This is called taking full responsibility. It is the underlying message of the Law of Kamma, and it may be Buddhism’s greatest contribution to the science of good governance.


\newpage

{\small\sffamily

\noindent\textbf{Ajahn Brahm}

\medskip

Venerable Ajahn Brahm (Brahmavamso) was born Peter Betts in London in 1951. In the late 1960s he won a scholarship to study theoretical physics at Cambridge University. After graduating, he taught in high school for one year before travelling to Thailand to become a monk.

Ajahn Brahm was ordained in Bangkok at the age of twenty-three by the Abbot of Wat Saket. He subsequently spent nine years studying and training in the forest tradition under Venerable Ajahn Chah (Phra Bodhinyana Mahathera).

In 1983, Ajahn Chah asked Ajahn Brahm to help establish a forest monastery near Perth, Western Australia. He is now the abbot of Bodhinyana Monastery, in Serpentine, Western Australia; Spiritual Director of the Buddhist Society of Western Australia; Spiritual Adviser to the Buddhist Society of Victoria; Spiritual Adviser to the Buddhist Society of South Australia; and Spiritual Patron of the Bodhikusuma Centre in Sydney.

In October 2004, Ajahn Brahm was awarded a John Curtin Medal for his vision, leadership and service to the Australian community by Curtin University.

On the occasion of the 60th anniversary of His Majesty the late King Bhumibol Adulyadej’s accession to the throne in 2006, Ajahn Brahm was granted the ecclesiastical title of Phra Visutisangvornthera on June 10, 2006.

Ajahn Brahm has authored the following books:

\begin{itemize}
\item \emph{Opening to The Door of Your Heart} or \emph{Who Ordered This Truckload of Dung?}
\item \emph{Mindfulness Bliss and Beyond} or \emph{Happiness Through Meditation}
\item \emph{Good? Bad? Who Knows?} or \emph{Don’t Worry, Be Grumpy}
\item \emph{The Art of Disappearing}
\item \emph{Kindfulness}
\item \emph{Bear Awareness}
\end{itemize}

\end{document}
